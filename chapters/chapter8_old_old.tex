\chapter{Conclusions and Future Work}
\label{chp:b8}

In this thesis, HDR image similarity problem is investigated through an experimental user study and two similarity models are proposed to model this subjective phenomenon. In the user experiment, the participants are asked to choose the image that is similar to the given reference image. The experiment is conducted through a web-based interface in a fashion that allows the participants to examine the images on different exposures. Which makes it possible to collect the responses of more than 1200 users that are reached through a crowdsourcing platform.

In order to gain insight about the subjective evaluation of HDR image similarity, the experiment data is analysed for different TMOs, image features and distance metrics. The selected TMOs are heavily used TMOs that prioritize different aspects of tone mapping. The image features include commonly used low level features as well as deep learning features. The distance metrics are chosen in a way that is suitable for the selected image feature representation. Correlations between human judgements and these quantitative features are computed to assess how much each feature contributes to visual similarity. Lastly, two combined features which perform better than individual image features are also proposed with the weights of contributing features are estimated from the user data.

Reliable assessment of image similarity lies at the hearth of many computer vision applications. In this thesis, an application is given to demonstrate how HDR image similarity can be used for consistently tone mapping various HDR images following a created style. This tone mapping method, namely style based tone mapping, estimates the tone mapping parameters for the given image from the tone mapping parameters of a set of manually tone mapped calibration images using the similarity between the given image and calibration images. Although a small number of calibration images used, the operator is shown to depict the given style and produce satisfying tone mapping results for HDR images that has different image statistics. 

It should be also noted that, one of the main benefits of the presented operator is once the calibration images are tone mapped, the new HDR images can be automatically tone mapped with the created style without any intervention such as manual search of optimal parameters. Which makes this operator a suitable choice for the applications where batch tone mapping of HDR images is necessary. Initially, this operator uses rather a basic image similarity model, but then this model is improved with two different approaches derived from the findings from the user experiment. The improvements yielded with better tone mapping results in terms of style imitation while keeping the image characteristics intact. 

The conducted image similarity experiment also has certain limitations and drawbacks. Firstly, it relies on crowdsourcing, which was necessary to reach a wider audience to collect as much as data possible but made it impossible to control the viewing conditions of the participants. Different results could have been obtained if the experiments were done in a laboratory environment with controlled display and lighting conditions.

Secondly, the participants compared the HDR images on standard monitors and used sliders to visualize different image regions that are visible on different exposures. Similarly, different results could have been obtained if participants viewed the images on an HDR display. 

Finally, in the experiment, the meaning of similarity is not given intentionally and left to the interpretation of the participants. To reduce this uncertainty, future studies may explicitly define what is meant by similarity such as object similarity, color similarity, indoor-outdoor similarity, time-of-day similarity, etc.

\section{Future Work}

Although image similarity is an extensively studied topic, HDR image similarity has not gained as much attention yet. To investigate this subjective phenomenon, further experiments which consider ranking and rating tasks as well as pairwise comparisons can be conducted. Also, the proposed models can be extended with different types of features. Evaluations may include DCNNs that are either fine-tuned or trained with HDR data from the ground up.

Given the large number of image quality datasets and subjective evaluations in the form of mean opinions scores (MOS), whether image quality and similarity correlate with each other in the context of HDR imaging can be investigated. Image saliency can also be taken into account for similarity judgements as it was found to improve performance in some other domains~\cite{amirkhani2019inpainted}. Perhaps most importantly, the effect of calibrated HDR images for image similarity and retrieval tasks can be studied. As objects are represented with their true luminances in calibrated data, this may simplify the similarity assessment between the images. Also, with emerging standards for HDR video streaming such as HDR10+ and Dolby Vision, the HDR video similarity problem will gain importance in near future.

Finally, in the recent years, HDR video tone mapping is gaining popularity and more and more studies are conducted to tone map videos coherently~\cite{eilertsen2017comparative}. Style-based tone mapping can be extended for video tone mapping, where the number of images are too many to pick tone mapping parameters manually and often different scenes needs to have the same look of the general style of the video. 
