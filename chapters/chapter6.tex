\chapter{Conclusions and Future Work}
\label{chp:b6}

We performed two user experiments followed by statistical
analyses to get an in-depth understanding of image similarity for HDR
images. 
We first collected a large number of human
similarity responses via crowdsourcing, and then evaluated several  image features with respect to the collected data.
Evaluation is performed both on individual features and on their combination. When combining features,  two models both of which using logistic regression are considered.  Although both models
are found to perform comparably, the second one permits direct one-to-one comparison, making 
it more suitable for practical applications. We show one such
application, namely style-based tone mapping that benefits from the
experimental findings.
Key observations obtained in our work are the following:
%
\begin{enumerate}
\item When properly tone mapped images are used as compared to using either original or linearly scaled HDR images, higher correlations with human responses are obtained.
\item Most tone mapping operators (TMOs) yield comparable performance.
\item Deeply learned features, in comparison to hand-crafted features, correlate better with the human responses.
\item Among hand-crafted features, GIST yields the highest correlation, followed by color, luminance, and texture.
\item All of the estimated correlations for the second experiment are higher in comparison to those for the first experiment.
\end{enumerate}

The first observation highlights the importance of using tone mapped data
for HDR image similarity. While tone mapping is a lossy process, it
brings the data to a more meaningful range for the computation of most
features. However, some features are less dependent on tone mapping. For
instance the texture feature represented by the histogram of oriented
gradients is found to produce about the same correlation regardless of
whether HDR or tone mapped data is used. This is followed by the color
feature represented by 2D chromaticity histogram. Among the hand-crafted
features the largest difference is  observed for luminance
when tone mapped data is used. This can be expected as
non-linear luminance compression often eliminates large gaps in
luminance histogram where little useful information is present.

Perhaps unexpectedly, the second observation suggests that TMOs perform comparably. 
Although there exists a large
number of TMO evaluation studies, we are not aware of any work that
compares TMOs for the task of HDR image similarity. The lowest
performing operator is found to be Pattanaik et
al.'s~\cite{pattanaik2000time} algorithm. It is, however, known that
this algorithm highly depends on calibrated input data and viewing
conditions as it tries to accurately model the human visual system.

As for the third observation, it is not surprising to find that 
features obtained from a DCNN~\cite{simon14} trained over a large image
dataset~\cite{russakovsky2015imagenet} outperform simple hand-crafted
features. Similar findings are reported by image retrieval studies
conducted for low dynamic images~\cite{wan2014deep,gordo2016deep}. For
HDR images, our findings indicate that deep features are mostly useful if the images are tone mapped to the 8-bit per color channel domain
first. This is also expected as the training data of DCNNs are comprised of such images.

The fourth observation indicates that the GIST descriptor surpasses the
texture, luminance, and color features for HDR image similarity. In
addition to outperforming them, in fact, it performs surprisingly
consistently across different processing types. Despite having a smaller
correlation with the user data than the deep features, it exhibits less
variability overall. This may be a desirable property as it appears to
be minimally affected by how an HDR image is processed.

Pertaining to our last observation, it can be argued that seeking multiple
consistent responses by the participants are important; not only for
developing a more reliable model but also for assessing the correlation
of different features with user responses. For instance, inspection of
Tables~\ref{TabFirst} and~\ref{TabSecond} reveals that while deep
features correlate better with the user responses, this difference is
clearly magnified for the second experiment. In other words, as the
experimental findings become more reliable the merits and drawbacks of
different features becomes more noticeable.

We believe that our work simply scratches the surface of the HDR
image similarity problem. The proposed models can be extended with different
types of features. Further experiments which consider ranking and rating
tasks as well as pairwise comparisons can be conducted. Evaluations may
include DCNNs that are either fine-tuned or trained with HDR data from
the ground up.