% CHAPTER 1
\chapter{Introduction}
\label{chp:b1}
With the developments in digital imaging, the means of producing digital images became widely available and this led to vast amount of images and diverse applications. Assessing visual similarity of images is an important task for various applications including image retrieval and indexing~\cite{liu2007survey}, classification and clustering~\cite{kleiman2016toward}, image editing and style transfer~\cite{rawat2018find}. Depending on the context of the application, image similarity might be defined differently and the developed techniques differ accordingly. Due to its importance, significant amount of research is dedicated to measuring image similarity. 

Unlike some computer vision tasks like depth estimation or object detection, image similarity does not have a ground truth. Typically, it is hard to estimate a certain degree of similarity between two images without a given context or other images to compare. The lack of ground truth makes the research of image similarity harder in two aspects, learning models from the ground truth data directly and measuring the performance of the proposed image similarity method reliably. This makes subjective experiments highly valuable for the image similarity research.

The need for High Dynamic Range (HDR) imaging was realized for the first time in computer graphics to deal with the requirements of physically accurate lighting simulation systems~\cite{glassner1995principles}. Such systems produced numerically unbounded pixel values, necessitating their storage in HDR file formats~\cite{ward1998rendering}. HDR images have typically been termed as ``scene-referred'' as opposed to ``display-referred'' -- a term used for Low Dynamic Range (LDR) images~\cite{Rein2010}. However, as display devices have traditionally been low dynamic range, displaying these images on LDR devices required an operation known as tone mapping~\cite{Tumb93,Ward97}. Numerous tone mapping operators (TMOs) have been developed in literature ranging from simple contrast adjustments to complex algorithms modeling the human visual system~\cite{Ferw96} and the properties of display devices~\cite{Mantiuk2008}. These operators are developed with different motivations and yielding with perceptually and statistically different images. Many methods have also been produced to create photographic HDR images of real-world scenes~\cite{Debe97}, including dynamic scenes~\cite{sen2012robust,kalantari2017deep}. Besides computer graphics, HDR imaging has many application areas including studying of fossils~\cite{theodor2009high}, cultural heritage and archaeology~\cite{happa2010high}, structural engineering~\cite{grinzato2009seismic}, architecture~\cite{cai2013high}, medical imaging~\cite{harifi2015efficient,rizzi2018visual}, forensics~\cite{brown2010forensic}, and automotive industry~\cite{wu2012fast}.

While the majority of the produced and stored images are standard images, HDR imaging offers to capture much wider luminance range, closer to the human eye, compared to standard LDR imaging. Currently, HDR video formats HDR10+~\cite{HDR10+} and Dolby Vision~\cite{chinnock2016dolby} are becoming the main HDR video standards and the number of HDR compatible commercial TVs, Blu-ray devices, gaming consoles and mobile devices supporting these video format standards is increasing. Besides, lately main streaming platforms like Amazon Prime Video and Netflix started serving HDR video content. These progress on the standardization and the developed devices will increase the demand in HDR content production very soon. Additionally, with the emerging technologies that require to exceed the human visual system limitations such as autonomous driving, HDR imaging gains more importance in safety critical computer vision applications.

%Until the HDR devices become the main standard, in order to display HDR images on standard displays, HDR images are mapped to low dynamic range, which is a process known as tone mapping. There are many tone mapping operators that    
%Despite the lack of native HDR sensors and commercially available HDR displays, the popularity of HDR imaging increases. 

Many research problems and applications are common to both HDR and LDR imaging. However, LDR images consist of 8 bit integers, whereas HDR images are represented with floating point data and if they have not been calibrated a single pixel intensity may refer to completely different illumination values on different scenes. Therefore, using the solutions and methods that are originally devised for standard LDR images may not be feasible for HDR images. 


\section{ Problem Definition}
Since image similarity is a perceptual phenomenon without ground truth, there are several studies that conducted human experiments. All these works, however, assume that the image is given in standard low dynamic range format where the brightness information is suitably quantized to match the dynamic range of traditional image display devices. Yet, growing number of applications utilize High Dynamic Range (HDR) images with unbounded brightness values. It may be argued that HDR images pose no extra challenges and approaches designed for LDR images may directly be used to assess visual similarity of HDR images as well. There are several counter-arguments, however. First, HDR images contain potentially uncalibrated floating point data and two images that have vastly different pixel values may actually be very similar to each other. Second, the richness of information in an HDR image, despite causing difficulties, may aid in similarity assessment. For example, pixel values corresponding to a bright light source can be much higher than that of a white reflecting surface in an HDR image, while the two objects are likely to map to similar intensities in an LDR image. Third, using a standard similarity measure for an HDR image requires tone mapping, a problem for which a multitude of algorithms, each with a number of parameters, exist~\cite{yeganeh2012objective}.

Hence, there is a need for investigating visual similarity for HDR images. This need is the motivation of this thesis where assessing visual similarity between two images is experimentally investigated. To this end, subjective human judgments using crowdsourcing is collected and features are evaluated by comparing them to human judgments. To our knowledge this thesis serves as the first rigorous attempt to evaluate how visual similarity can be assessed between HDR images. Using the findings from the perceptual study, a tone mapping methodology is proposed where tone mapping parameters are automatically computed to impart a certain user-defined style to a given HDR image using the similarity between this image and several calibration images that are used to create this style.

\section{Contributions and Outline of the Thesis}
The biggest motivation of this thesis is to investigate visual similarity for HDR images and demonstrate an application that benefits from these investigations. In this regard, the main contributions of this thesis can be listed as below: 
\begin{itemize}
    \item A tone mapping methodology, namely style-based tone mapping, that uses image similarity to automatically estimate the tone mapping parameters from the tone mapping parameters of a set of given images to follow a certain style,
    \item A novel web-based interface to conduct a perceptual similarity experiment, and the assessment data collected through crowdsourcing,
    \item Analysis on the experiment data, in the sense of the agreement of various image features and the corresponding distance metrics to the user responses, besides examining the effect of tone mapping operators,
    \item A similarity model that consists of combination of image features with contribution of each feature estimated by user experiment data,
    \item Two different methodologies to improve the presented tone mapping operator by employing the findings obtained by user experiment.
\end{itemize}


The organization of the rest of this thesis is as follows, in Chapter~\ref{chp:b2}, a review of HDR imaging and image similarity is given. The image features and the corresponding distance metrics used in the rest of the thesis are given in Chapter~\ref{chp:b3}. Style-based tone mapping operator is presented in detail in Chapter~\ref{chp:b4} with some example styles replicated with the developed tool and the tone mapping results following these styles. Due to the subjective nature of the image similarity problem, an experimental study that assesses visual similarity between HDR images is conducted. The experimental setup and the details about data gathering is given in Chapter~\ref{chp:b5}. The analyses on the user experiment such as the correlation of the image features and the corresponding distance metrics to the user responses, and the effect of the tone mapping operators to these analyses are given in this chapter. In addition, two combined models that are estimated using experiment data are also given. In Chapter~\ref{chp:b6}, two methods to improve the style-based tone mapping operator by using the results obtained through user experiment are presented. 

Lastly, discussion about the findings and the results are given in Chapter~\ref{chp:b7} and the summary of the thesis, several limitations and future directions are given in Chapter~\ref{chp:b8}.






