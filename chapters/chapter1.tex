% CHAPTER 1
\chapter{Introduction}
\label{chp:b1}

%paper_start
Assessing visual similarity of images is an important task for various applications including  image retrieval and indexing~\cite{liu2007survey}, classification and clustering~\cite{kleiman2016toward}, image editing and style transfer~\cite{rawat2018find}. Due to its importance, significant amount of research is dedicated to measuring image similarity. Because visual similarity is a perceptual phenomenon without ground truth, several human experiments are conducted as well. All these works, however, assume that the image is given in standard low dynamic range format where the brightness information is suitably quantized to match the dynamic range of traditional image display devices. Yet, growing number  of applications utilize High Dynamic Range (HDR) images with unbounded brightness values. It may be argued that HDR images pose no extra challenges and approaches designed for LDR images may directly be used to assess visual similarity of HDR images as well.  There are several counter-arguments, however. First, HDR images contain potentially uncalibrated floating point data and two images that have vastly different pixel values may actually be very similar to each other. Second, the richness of information in an HDR image, despite causing difficulties, may  aid in similarity assessment. For example, pixel values corresponding to a bright light source can be much higher than that of a white reflecting surface in an HDR image, while the two objects are likely to map to similar intensities in an LDR image. Third using a standard similarity measure for an HDR image requires tone mapping, a problem for which a multitude of algorithms, each with a number of parameters, exist~\cite{yeganeh2012objective}.

Hence, there is a need for investigating visual similarity for HDR images. This need is the motivation of the present work where we experimentally investigate assessing visual similarity between two images. To this end, we collect subjective human judgments using crowdsourcing and evaluate features by comparing them to human judgments. Our data collection via crowdsourcing is performed in two stages.  In the first stage, $100$ HDR images from several HDR image data\-bases are used in a pairwise assessment task. Due to a large number of combinations, this phase primarily serves as an unbiased exploration of a large search space. In the second stage, we focus on the already tested images from the first phase to collect multiple responses for each test case.

Our experimental control factors include choice of tone mapping operator, choice of distance metric, and choice of image feature. We use commonly used low-level image features such as color, luminance, and texture histograms; advanced features such as GIST~\cite{oliva2001modeling};  deeply learned features~\cite{simonyan2014very}; and combined features estimated using logistic regression. To our knowledge our work serves as the first rigorous attempt to evaluate how visual similarity can be assessed between HDR images. Using our findings, we propose a tone mapping methodology where tone mapping parameters are automatically computed to impart a certain user-defined style to a given HDR image using the similarity between this image and several calibration images that are used to create this style.
%paper_end
\section{ Problem Definition}


\section{Contribution}


%\section{The Outline of the Thesis}




