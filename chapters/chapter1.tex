% CHAPTER 1
\chapter{introduction}
\label{chp:b1}
With the developments in digital imaging, the means of producing digital image became widely available and this led to vast amount of images and diverse applications. Assessing visual similarity of images is an important task for various applications including image retrieval and indexing~\cite{liu2007survey}, classification and clustering~\cite{kleiman2016toward}, image editing and style transfer~\cite{rawat2018find}. Depending on the application, image similarity might be defined differently and the developed techniques differs accordingly. Due to its importance, significant amount of research is dedicated to measuring image similarity. 

Unlike some computer vision tasks like depth estimation or object detection, image similarity does not have ground truth. Typically, it is hard to estimate a certain degree of similarity between two images without given context or other images to compare. The lack of ground truth makes the research of image similarity harder in two aspects, learning models from the ground truth data directly and measuring the performance of the proposed image similarity method reliably. This makes subjective experiments highly valuable for the image similarity research.

While the majority of produced and stored images are standard images, High Dynamic Range (HDR) imaging offers to capture much wider luminance range, closer to the human eye, compared to standard Low Dynamic Range (LDR) imaging. Despite the lack of native HDR sensors and commercially available HDR displays, the popularity of HDR imaging increases. In order to display HDR images on standard displays, HDR images are mapped to low dynamic range, which is known as tone mapping. There are many tone mapping operators that are developed with different motivations and yielding with different looks and different image statistics. Comparison and subjective evaluation of tone mapping operators is also an active research area~\cite{kundu2017large, krasula2016preference}.

Whereas LDR images consist of 8 bit integers, HDR images are represented with floating point data and if they have not been calibrated a single pixel intensity may refer to completely different illumination values on different scenes. Therefore, using the solutions and methods that are originally devised for standard LDR images may not be feasible for HDR images.


\section{ Problem Definition}
Since image similarity is a perceptual phenomenon without ground truth, there are several studies that conducted human experiments. All these works, however, assume that the image is given in standard low dynamic range format where the brightness information is suitably quantized to match the dynamic range of traditional image display devices. Yet, growing number of applications utilize High Dynamic Range (HDR) images with unbounded brightness values. It may be argued that HDR images pose no extra challenges and approaches designed for LDR images may directly be used to assess visual similarity of HDR images as well. There are several counter-arguments, however. First, HDR images contain potentially uncalibrated floating point data and two images that have vastly different pixel values may actually be very similar to each other. Second, the richness of information in an HDR image, despite causing difficulties, may aid in similarity assessment. For example, pixel values corresponding to a bright light source can be much higher than that of a white reflecting surface in an HDR image, while the two objects are likely to map to similar intensities in an LDR image. Third, using a standard similarity measure for an HDR image requires tone mapping, a problem for which a multitude of algorithms, each with a number of parameters, exist~\cite{yeganeh2012objective}.

Hence, there is a need for investigating visual similarity for HDR images. This need is the motivation of this study where assessing visual similarity between two images is experimentally investigated. To this end, subjective human judgments using crowdsourcing is collected and features are evaluated by comparing them to human judgments. To our knowledge this study serves as the first rigorous attempt to evaluate how visual similarity can be assessed between HDR images. Using the findings from the perceptual study, a tone mapping methodology is proposed where tone mapping parameters are automatically computed to impart a certain user-defined style to a given HDR image using the similarity between this image and several calibration images that are used to create this style.

\section{Contributions and Outline of the Thesis}
The main motivation of this thesis is to investigate visual similarity for HDR images. In Chapter~\ref{chp:b2}, a review of image similarity and HDR imaging is given as well as the image features distance metrics used in the rest of the study.

Due to the subjective nature of the image similarity problem, an experimental study that assesses visual similarity between HDR images is conducted. The novel web-based interface for the experiment, experimental setup and the details about data gathering through crowdsourcing is given in Chapter~\ref{chp:b3}.

Next, the agreement of various image features and the corresponding distance metrics to the collected user data is investigated. In addition, a combined feature is estimated using experiment data. The effect of tone mapping operator is also presented by repeating the analysis on the images that tone mapped with commonly used tone mapping operators. In Chapter~\ref{chp:b4}, these analyses on the user experiment are given.

A tone mapping methodology, namely style based tone mapping, that uses image similarity to automatically estimate tone mapping parameters from the tone mapping parameters of a set of given images to follow a certain style is given in Chapter~\ref{chp:b5}. The improvements achieved by using the findings from user study is also provided.
Lastly, a summary of the study, several limitations and future directions are given in Chapter~\ref{chp:b6}.






