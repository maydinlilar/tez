\documentclass[chaparabic,ceng,phd,12pt,oneandhalf,fivejury]{metu}
\usepackage{appendix}
\usepackage{longtable}
\usepackage[pdftex]{hyperref}
\usepackage[all]{hypcap}
\usepackage{todonotes}
\usepackage{graphicx}
\graphicspath{ {./images/} }
\usepackage[figuresright]{rotating}
\usepackage{xy} 
\usepackage{booktabs}
\usepackage{pifont}
\usepackage{color}
\usepackage{listings}
\usepackage{pdfpages}
\usepackage{array}
\usepackage{algorithm}
\usepackage{algorithmic}
\usepackage{float}
\usepackage{caption}
\usepackage{lastpage}
\usepackage{afterpage}
\usepackage{lipsum}
\usepackage{adjustbox}
\usepackage{rotating}
%
\usepackage{pdflscape}

% \usepackage{graphicx}
\usepackage{amsmath,amssymb} % define this before the line numbering.
% \usepackage{ruler}
\usepackage{color}
% \usepackage{cite}
% \usepackage[utf8x]{inputenc}
% \usepackage{footnote}
% \makesavenoteenv{tabular}
% \makesavenoteenv{table}

\renewcommand{\sectionautorefname}{\S}
\renewcommand{\subsectionautorefname}{\S}

\newcommand{\norm}[1]{\left\lVert#1\right\rVert}

\captionsetup{belowskip=12pt,aboveskip=8pt}
\newcommand{\tab}{\hspace*{2em}}
\DeclareGraphicsExtensions{.pdf,.png,.jpg}


\usepackage{amsmath}
\usepackage{siunitx}
\usepackage{textcomp}
\usepackage{subcaption}


\usepackage{tikz}
\usepackage{mathtools}
% \usepackage{rotating}
%\PassOptionsToPackage{figuresright}{rotating}

\DeclarePairedDelimiter\ceil{\lceil}{\rceil}
\DeclarePairedDelimiter\floor{\lfloor}{\rfloor}


\newcommand{\EA}[1]{\textcolor{red}{[EA: #1]}}

% Name and Surname
\author{Merve Aydınlılar}
% Thesis Title English and Turkish
\title{Visual Similarity for HDR Images with Applications to Tone Mapping}
\turkishtitle{ODTÜ Tez Şablonu}

\date{February 2021}
 
% prof : Prof. Dr.
% assocprof : Assoc. Prof. Dr.
% assistprof : Assist. Prof. Dr.
% dr : Dr.
%
% Director of Institute
\director[prof]{Halil Kalıpçılar}
% Head of Department
\headofdept[prof]{Halit Oğuztüzün}
%
% Supervisor : English and Turkish
\supervisor[prof]{Ahmet Oğuz Akyüz}
% \turkishsupervisor{  } %if you will hard-code the academic title
%
% Affiliation of Supervisor in English and possibly in Turkish
\departmentofsupervisor{Computer Engineering, METU}

\cosupervisor[prof]{Sibel Tarı}
\departmentofcosupervisor{Computer Engineering, METU}
%
% Committee Members
% In general members are sorted according to their academic titles
%
% Proffesors (1)
% Associate Professors (2)
% Assistant Professors (3)
% Other (4)
% 
% IMPORTANT:  All affiliatons should fit in a single line
% If affiliation line is broken into two lines you should shorten the affiliation by using 
% abbrevations or any other means
%
% First committee member should be the chair of examining committee
% Typically the chair is one of the highest ranked committee members
% Ask your supervisor if you are not sure
\committeememberi[assistprof]{Name Surname}
\affiliationi{Computer Engineering, METU}
% Second committee member is always your supervisor
\committeememberii[prof]{Ahmet Oğuz Akyüz}
\affiliationii{Computer Engineering, METU}

% IMPORTANT: If you are Ph.D. student your co-supervisor can not be in your 
% examination committee.

% \def\@proftitlename{Prof. Dr.}\def\@tproftitlename{Prof. Dr.}
% \def\@assocproftitlename{Assoc. Prof. Dr.}\def\@tassocproftitlename{Doç. Dr.}
% \def\@assistproftitlename{Assist. Prof. Dr.}\def\@tassistproftitlename{Yrd. Doç. Dr.}
% \def\@drtitlename{Dr.}\def\@tdrtitlename{Dr.}

\committeememberiii[prof]{Mine Özkar}
\affiliationiii{Architecture, ITU}
% Fourth committee member
\committeememberiv[prof]{Tolga Kurtuluş Çapın}
\affiliationiv{Computer Engineering, TED University}
% Fifth committee member
\committeememberv[assocprof]{Sinan Kalkan}
\affiliationv{Computer Engineering, METU}
%
% Keywords : English & Turkish, Comma seperated
\keywords{keywords, seperated, with, comma}
\anahtarklm{virgülle, ayrılmış, anahtar, kelimeler}
%
% Abstract in English
%
\abstract{
Assessing visual similarity for low dynamic range (LDR) images is an extensively studied problem. Promising results are obtained when an image database is queried to find the images containing a specific object of interest, a task known as exact matching. However, this problem becomes more difficult if one needs approximate similarity, which is the task of determining images that are merely visual similar to a query image. In this study, the aim ıs to shed light on this problem with an experimental approach. To this end, first crowd-sourcing data is collected through a novel web-based experimental interface, in which the participants assess the visual similarity of HDR images. Then, a set of features collected to assess how well they correlate with the participants responses. Features computed by convolutional neural networks are also included, which are proven to be successful in standard similarity tasks. The correlation of each feature with the crowdsourcing data is computed, first in isolation, and then merging multiple features by using different weights for each feature determined through a metric learning approach. Based on the obtained results, a style-based tone mapping algorithm is proposed that can successfully impart a user-defined style to various HDR images determined to be similar with respect to the proposed metric.
}
%
% Turkish Abstract
%
\oz{
Çok çeşitli amaçlarla kullanılmak üzere standard imgelerin benzerliğini belirleyen bir çok algoritma bulunmasına rağmen YDA imgeler için yapılmış çalışma sayısı oldukça azdır. Teknolojinin gelişmesiyle birlikte YDA imgelerin benzerliğinin objektif bir biçimde değerlendirilmesi önem kazanacaktır. Bu sebeple, imgelerin benzerliklerini ölçmede etkili olacağını düşündüğümüz, imgelerin alt ve orta seviye özelliklerini ve uzaklık metriklerini öne sürüyoruz. Yaptığımız kullanıcı deneyi ile subjektif benzerlik skorları elde edilmiştir ve önerilen özelliklerin bu skorlarla olan korelasyonu değerlendirilmiştir. Daha sonra, bu özelliklerin çeşitli ağırlıklarla oluşturulan kombinasyonunun her bir özellikle tek tek elde edilen sonuçlardan daha başarılı sonuç verdiği gözlenmiştir. Geliştirilen tekniğin bir uygulaması olarak YDA imgelerin standard imgelere dönüştürülmesi probleminde imgeler arası benzerliğin kullanılmasının etkisi gösterilmiştir. 
} 
%
% Dedication 
\dedication{Lorem ipsum dolor sit amet}
%
%
% Acknowledgements   
\acknowledgments{}

%
% End of Personal and Introductory  Information
%%%%%%%%%%%%%%%%%%%%%%%%%%%%%%%%%5
\begin{document}
% Preliminaries
\begin{preliminaries}
% If you are willing to use any custom stuff before Chapters, put it here
% Such as List of Abbreviations
% Check the abbreviations.tex for a template list of abbreviations

\begin{theglossary}{LONGESTABBRV}

\item[1D] 1 Dimensional 
\item[2D] 2 Dimensional
\item[2AFC] 2 Alternative Forced Choice
\item[BF] Bileteral Filter
\item[CNN] Convolutional Neural Network
\item[DCNN] Deep Convolutional Neural Network
\item[EMD] Earth Mover's Distance
\item[HDR] High Dynamic Range
\item[HSV] Hue, Saturation, Value
\item[HVS] Human Visual System
\item[LDR] Low Dynamic Range
\item[MOS] Mean Opinion Scores
\item[SIFT] Scale-Invariant Feature Transform
\item[SURF] Speeded Up Robust Features
\item[TMO] Tone Mapping Operator
\item[UM] Unsharp Masking

\end{theglossary}

% End of Preliminaries
\end{preliminaries}
%   
% Latex content Goes Here 
% 
%

\setlength{\parindent}{0em}
\setlength{\parskip}{10pt}

% You can add as many chapters
% CHAPTER 1
\chapter{Introduction}
\label{chp:b1}

%paper_start
Assessing visual similarity of images is an important task for various applications including image retrieval and indexing [28], classification and clustering [24], image editing and style transfer [42]. Due to its importance, significant amount of research is dedicated to measuring image similarity. Because visual similarity is a perceptual phenomenon without ground truth, several human experiments are conducted as well. All these works, however, assume that the image is given in standard low dynamic range format where the brightness information is suitably quantized to match the dynamic range of traditional image display devices. Yet, growing number of applications utilize High Dynamic Range (HDR) images with unbounded brightness values. It may be argued that HDR images pose no extra challenges and approaches designed for LDR images may directly be used to assess visual similarity of HDR images as well. There are several counter-arguments, however.

First, HDR images contain potentially uncalibrated floating point data and two images that have vastly different pixel values may actually be very similar to each other. Second, the richness of information in an HDR image, despite causing difficulties, may aid in similarity assessment. For example, pixel values corresponding to
a bright light source can be much higher than that of a white reflecting surface in an HDR image, while the two objects are likely to map to similar intensities in an LDR image. Third using a standard similarity measure for an HDR image requires tone mapping, a problem for which a multitude of algorithms, each with a number of parameters, exist [60].

Hence, there is a need for investigating visual similarity for HDR images. This need is the motivation of the present work where we experimentally investigate assessing visual similarity between two images. To this end, we collect subjective human judgments using crowdsourcing and evaluate features by comparing them to human judgments. Our data collection via crowdsourcing is performed in two stages. In the first stage, 100 HDR images from several HDR image databases are used in a pairwise assessment task. Due to a large number of combinations, this phase primarily serves as an unbiased exploration of a large search space. In the second stage, we focus on the already tested images from the first phase to collect multiple responses for each test case.


Our experimental control factors include choice of tone mapping operator, choice of distance metric, and choice of image feature. We use commonly used low-level image features such as color, luminance, and texture histograms; advanced features such as GIST [39]; deeply learned features [53]; and combined features estimated using logistic regression. To our knowledge our work serves as the first rigorous attempt to evaluate how visual similarity can be assessed between HDR images. Using our findings, we propose a tone mapping methodology where tone mapping parameters are automatically computed to impart a certain user-defined style to a given HDR image using the similarity between this image and several calibration images that are used to create this style.
%paper_end
\section{ Problem Definition}


\section{Contribution}


%\section{The Outline of the Thesis}





\chapter{Related Work}
\label{chp:b2}

\section{Image Similarity}
%paper_start
Traditionally, image similarity is measured by measuring the distance between hand crafted features extracted from each image. These hand crafted features include simple descriptors such as color/luminance histograms, or improved ideas, including histogram of oriented gradients~\cite{dalal2005histograms}. GIST~\cite{oliva2001modeling}, SIFT~\cite{lowe2004distinctive}, SURF~\cite{bay2006surf}. These features are compared using several types of distance metrics. Recently, deep convolutional neural networks (DCNNs) became the state of art for image classification. Starting with AlexNet~\cite{krizhevsky2012imagenet} and followed by deeper networks such as VGG~\cite{simonyan2014very}, GoogLeNet\cite{szegedy2015going}, and ResNet~\cite{he2016deep}, DCNNs started to perform near human level success for image classification. Their success lead to use feature vectors that have been obtained from DCNNs for image retrieval~\cite{wan2014deep,gordo2016deep}. Unlike previous approaches that are based on hand-crafted features, DCNNs learn the feature vector itself directly from the image. One major drawback of using
DCNNs is the need for using very large labeled datasets for training, which is difficult to obtain or not available at all for most problem domains. Transfer learning~\cite{yosinski2014transferable} aims to solve this problem by using pretrained networks on large scale datasets such as ImageNet~\cite{russakovsky2015imagenet}. The basic method is to give the images to the pre-trained network and use the output of the last fully connected layers as feature vectors~\cite{donahue2014decaf,wan2014deep} -- an approach that we also adopt in our work.

Visual similarity is a perceptual phenomenon without ground-truth data. This makes collecting data using crowdsourcing experiments valuable. Indeed, there are several crowdsourcing-based works~\cite{lun2015elements,saleh2015learning,kleiman2016toward} that address shape or style similarity problems and conduct user experiments to either derive or validate models.

Of most related to our work are two similarity studies that also employ subjective experiments. Among these, in Rogowitz et al.~\cite{rogowitz1998perceptual}, human participants are asked to judge image similarity using two different experiments: one involving printouts of images (called table scaling) and the other using a computer based comparison (called computer scaling). These results are compared with computational similarity approaches~\cite{frese1997methodology} and simple CIELAB histograms. It was found that both table and computer scaling yield similar results and color is a major factor influencing similarity for human observers.

In another study~\cite{neumann2006image}, user experiments are conducted to evaluate the relationship between an image-indexing system and perceived similarity in an LDR setting. The tested image indexing system is based on basic properties of early stages of human vision -- chromaticity, luminance, and texture. Two-alternative forced-choice (2AFC) method is used for all experiments. Three images are shown to the observer, the query image and two test images. Of these two images one image is called the target and the other the distractor. These images are selected based on the rankings obtained from the image-indexing system. Then the correlation between the users' preference and index rank is investigated. First, each index, chromaticity, luminance, and texture are calculated separately. From these indexes chromaticity is found to give the best results. Then for the second experiment, combinations of the indexes are evaluated. The combination of chromaticity and texture indices are found to give better results than chromaticity alone and the combination of all indices are found to give the best result.
%paper_end

\section{HDR Imaging}
The need for HDR imaging was realized for the first time in computer graphics to deal with the requirements of physically accurate lighting simulation systems~\cite{glassner1995principles}. Such systems produced numerically unbounded pixel values, necessitating their storage in HDR file formats~\cite{ward1998rendering}. HDR images have typically been termed as ``scene-referred'' as opposed to ``display-referred'' -- a term used for LDR images~\cite{Rein2010}. However, as display devices have traditionally been low dynamic range, displaying these images on LDR devices required an operation known as tone mapping~\cite{Tumb93,Ward97}. Numerous tone mapping operators (TMOs) have been developed in literature ranging from simple contrast adjustments to complex algorithms modeling the human visual system~\cite{Ferw96} and the properties of display devices~\cite{Mantiuk2008}. Many methods have also been produced to create photographic HDR images of real-world scenes~\cite{Debe97}, including dynamic scenes~\cite{sen2012robust,kalantari2017deep}.

While HDR imaging has always been an active field of research, recent developments in HDR imaging~\cite{Rein2010,Banterle2011,chalmers2016high}, in particular those pertaining to HDR image and video capture~\cite{tocci2011versatile,froehlich2014creating} and display systems~\cite{seetzen2004high} to allow direct rendition of HDR images will likely make HDR images more ubiquitous in the near future. Consequently, we believe that assessing visual similarity of HDR images will also become an important problem.

As mentioned above, although visual image similarity is an extensively studied subject~\cite{liu2007survey}, to our knowledge there is no study that directly addresses this problem for HDR images. Thus, understanding the nature of image similarity for HDR images and developing an objective similarity measure is the primary goal of this paper. 
%paper_end

\subsection{Tonemapping}

\section{Features}
\label{sec:features}
%paper
In this study, five kinds of features are used to model HDR images: color, luminance, texture, GIST, and DCNN features. Table \ref{tab:table_feature} lists these features together with their representations and the distance metric used for each feature. The following sections outline the details of these features and the corresponding distance metrics.

\begin{table}[h!]
\caption{HDR Image features and distances}
\centering
\begin{tabular}{c|c|c}
\label{tab:table_feature}
\textbf{Feature} & \textbf{Model} & \textbf{Distance Metric}\\
\hline
Color  & 2D chromaticity histogram & Earth Mover's Distance (EMD) \\
Luminance  & 1D (relative) luminance histogram & EMD \\
Texture  & Histograms of gradients & EMD \\
GIST  & Feature vector & Cosine distance \\
VGG16/VGG19 - fc6 & Fused fc6 layer & Cosine distance  \\
VGG16/VGG19 - fc7 & Fused fc7 layer & Cosine distance
\end{tabular}
\end{table}

%paper
\subsection{Color}
%paper
Since the early days of the image similarity research, color has been used as one of the most discriminative cues~\cite{neumann2006image}. In this study,  we used the $a$ and $b$ channels of the CIELAB color space~\cite{iso201111664} to represent chromaticity information. This is an opponent color space, where the $a$ channel represents red/green opponent colors and the $b$ channel yellow/blue opponent colors. We used a 2D chromaticity histogram to represent the distribution of colors in a given image. Each dimension contained $15$ bins for a total of $225$ bins.  Figure~\ref{fig:hists} shows this histogram for the Mason Lake image from the
dataset~\cite{fairchild2007hdr}.

\begin{figure}
\centering
\caption{Sample image (left), 2D histogram (right).}
\label{fig:hists}
\begin{tabular}{c c}
\includegraphics[height=1.8in]{figures/chapter2/MasonLake.jpg} &
\includegraphics[height=1.8in]{figures/chapter2/57_histab.png}

\end{tabular}
\end{figure}

%paper
\subsection{Texture}
%paper
Texture is the second most used feature for content based image retrieval systems after chromatic features.  This feature is especially helpful for discriminating images that have similar color but different spatial characteristics such as blue sky and sea or sand and buildings.  To represent the texture information we used histogram of gradient magnitudes~\cite{sharma2015histogram}.
%paper
\subsection{Luminance}
%paper
The main difference between an HDR and an LDR image is the much wider range of luminance distribution for the former. A single HDR image may contain very low luminances corresponding to highly shadowed regions as well as very high luminances corresponding to bright highlights. Therefore, we hypothesized that the luminance distribution of an HDR image may be an important cue for visual similarity. The luminance distribution is modeled using a 1D (relative) luminance histogram with $50$ bins.
%paper
\subsection{GIST Features}
%paper
The GIST descriptor~\cite{oliva2001modeling} aims to represent the dominant spatial structure of a scene by using low level multi-scale representations. This descriptor defines the scene as a whole rather than focusing on individual objects or regions. Discriminative properties of a scene are listed as naturalness, openness, roughness, expansion, and ruggedness. The class of a scene, e.g., man-made, natural, indoor, outdoor, etc., is determined by these properties.

The procedure for extracting GIST descriptors consists of applying Gabor filters that are scaled and orientated differently to the input image, dividing the filter response map into a grid in order to have spatial information, averaging the filter response in each grid, and concatenating the results to obtain the final feature vector, i.e. the GIST descriptor.

%paper
\subsection{Deeply Learned Features}
%paper
Recently, DCNNs have started to dominate object recognition and image classification tasks, achieving near human success rates~\cite{krizhevsky2012imagenet,simon14,zhou2017scene}. These models are trained with large prelabeled datasets and develop a hierarchical model that becomes more aware of the content of the image rather than the underlying pixel values. To our knowledge currently there is no DCNN model that is trained on HDR images for the purpose of image indexing, scene classification, or visual similarity tasks.
Furthermore, there is no prelabeled large HDR image dataset to use for training a DCNN model from scratch. Therefore in this study, we used transfer learning method to employ pretrained DCNNs for our perceptual
similarity problem. 

For feature extraction, pretrained AlexNet~\cite{krizhevsky2012imagenet} and two variants of VGG networks, VGG16 and VGG19, are used~\cite{simon14}. All networks are trained on the ImageNet~\cite{russakovsky2015imagenet} dataset, but we also evaluated their performance when trained using
different datasets. For transfer learning, the last fully connected layer, which contains classification outputs, is removed and the remaining $4096$ dimensional two fully connected layers, \textbf{fc6} and \textbf{fc7}, are used as feature vectors. As suggested by Simonyan and Zisserman~\cite{simonyan2014very}, the results obtained from VGG16 and VGG19 are fused (by taking an average) and it is observed that
the fused version performs better than both VGG16 and VGG19. The distance between the feature vectors are calculated using cosine distance, which is a commonly used distance metric for deep learning features. 
%paper
\section{Distance Metrics}
%paper
The use of a proper distance metric is as important as the features themselves. Each feature representation
may require a different distance metric. In this section, we briefly describe the definitions and properties of the dissimilarity measures that we used for different types of features.
%paper
\subsection{Euclidean Distance}
%paper
The Euclidean distance between two histograms p and q is calculated as:
\begin{equation}
dist_{euc}(p,q) = \sqrt{\sum_i(p_i-q_i)^2},
\end{equation}
where i is the bin index. In general, dissimilarity obtained by Euclidean distance for histograms is not satisfactory as it does not take bin proximity into account.
%paper
\subsection{Bhattacharyya Distance}
%paper
Bhattacharyya distance~\cite{bhattacharyya1946measure} measures the overlap between two distributions. If p and q are two histograms, it can be calculated as:
\begin{equation}
dist_{bhat}(p,q) = -\ln \left( \sum_i \sqrt{p_i.q_i} \right).
\end{equation}

For our HDR similarity problem Bhattacharyya distance gives slightly better results than Euclidean distance. However, it also suffers from the same problem that the proximity of the bins is not taken into account.
%paper
\subsection{Earth Mover’s Distance}
%paper
Earth Mover’s Distance (EMD) is a dissimilarity metric commonly used for image the retrieval problems~\cite{rubner2000earth}. EMD aims to capture the perceptual similarity between two distributions by calculating the minimal cost of transforming one distribution to the other. Unlike the other dissimilarity metrics, EMD can be calculated for varying-size partitions of the data, called signatures.
Signatures consist of dominant clusters of the data, represented as $si = (m_i, w_i)$ pairs where mi is the cluster center and $w_i$ is the size of the cluster. EMD does not require the signatures to have the same number of clusters – ground distances between cluster centers are sufficient. Histograms are signatures with bin centers corresponding to cluster centers, mi, and normalized bin values to weights, $w_i$.

The total amount of work to transform distribution
p to q with flow f is:
\begin{equation}
WORK(P,Q,F) = \sum_i^m \sum_j^n d_{ij}f_{ij}, 
\end{equation}
where dij is the ground distance between cluster centers i and j. The optimal flow f that results with the minimum work, can be found by any linear optimization algorithm. When f is calculated, the EMD between p and q is defined as:
\begin{equation}
EMD(p,q) = {{\sum \sum d_{ij}f_{ij}}\over{\sum \sum f_{ij}}}.
\end{equation}
In our problem, bin centers correspond to color values
(ab values in the CIELAB space) and ground distances
are calculated as Euclidean because of the perceptual
uniformity of the CIELAB color space.

Figure \ref{fig:sim_comp} compares the effect of these three distance
metrics for a sample image from the dataset. The image
on the first column is the query image, and in each row,
the most similar five images from the dataset are shown.
The distance metric used in first row is Euclidean, the
second row is Bhattacharyya, and the last row is the
EMD. It can be argued that more similar images are
found using the EMD metric.

\begin{figure} 
\centering
\caption{A comparison of dissimilarity metrics for histogram-based features. The leftmost image is the query image, the most similar five images from the dataset are shown in each row: Euclidean distance (first row), Bhattacharyya distance (second row), Earth Mover’s distance (third row).}
\label{fig:sim_comp}
\includegraphics[width=\textwidth]{figures/chapter2/16sims.png}
\vspace{10pt}
\end{figure}

%paper
\subsection{Cosine Similarity}
%paper
Cosine distance between two vectors $p$ and $q$ is calculated as:
\begin{equation}
dist_{cosine}(p,q) = 1 - {{\sum_{i=1}^{n}p_{i}q_{i}} \over {\sqrt{\sum_{i=1}^{n}p_{i}^2}} \sqrt{\sum_{i=1}^{n}q_{i}^2}} 
\end{equation}
Cosine distance is a widely used distance metric for deep representations. In this study, we used cosine distance for calculating the distances between CNN feature vectors and GIST features.
%paper

























\chapter{User Study}
\label{chp:b3}

\section{Dataset}

The set of images used in visual similarity experiments should be sufficiently diverse. Although such datasets exists for LDR images, there is no specific similarity dataset for HDR images. However, there exists HDR image datasets that were created for various purposes and by different authors. We therefore decided to select 100 HDR images from various such sources to present observers with a diverse set of images2. The used datasets were: Fairchild’s HDR Photographic Survey~\cite{fairchild2007hdr}, HDR-Eye~\cite{nemoto2015visual}, DEIMOS~\cite{klima2011deimos}, Empa HDR Image Database~\cite{EmpaHDR}, and pfstools HDR Image Gallery ~\cite{mantiuk2007high}. Thumbnails for the used images are shown in Figure \ref{fig:dataset}.

\begin{figure}
\begin{center}
\includegraphics[width=\textwidth]{figures/chapter3/dataset.png}
\caption{HDR images used in the visual similarity experiments.}
\label{fig:dataset}
\end{center}
\end{figure}


\section{Experiment Setting}
%paper
To measure perceptual similarity between HDR images, we conducted a 2AFC experiment. The experiment is publicly available\footnote{http://user.ceng.metu.edu.tr/~merve/userstudy/}. As we needed a large number of responses, we designed a web-based interface to collect crowdsourcing data. We used the HDRHTML technique~\cite{mantiuk2009visualizing} for visualizing HDR images on web browsers.

This technique uses a windowing approach to select a desired exposure range from the HDR image. Multiple exposures are encoded by combining a small set of basis images with opacity coefficients. The tone-curves of these basis images are approximated as a piece-wise linear function. Instead of finding optimal tone-curves and opacity coefficients, HDRHTML uses a precomputed optimal solution and use these tone-curve points and opacity coefficients for all images for fast processing. After basis images are created using the tone-curves, these basis images are used to reconstruct multiple exposures and gives user the control over exposure settings with a slider. By dynamically adjusting the position of the slider, the user can efficiently view the entire exposure range contained within the HDR image. These sliders are normally overlayed with the image histogram. We removed this overlay to prevent the image histogram from affecting the observers’ decisions. Figure \ref{fig:experiment} shows a sample trial from the experiment. An HDR reference image was shown at the top and two HDR test images were shown at the bottom. The sliders, which were mandatory to be adjusted, allowed all images to be inspected at different exposure levels.

\begin{figure}
\begin{center}
\includegraphics[width=\textwidth]{figures/chapter3/experiment.png}
\caption{EA sample trial from the experiment. The observers were asked to choose the most similar image to the reference image (top) from the test images (bottom). All images could be examined at different exposure levels by adjusting their sliders.
%paper
}
\label{fig:experiment}
\end{center}
\end{figure}

In each experimental session, 33 such image triplets were displayed to the observers. Thus, an experimental session consisted of 33 trials. In each trial, the observers were asked to choose which of the two test images was visually more similar to the reference image. Here it is important to note that we did not ask users to decide for a specific type of similarity such as object, color, etc. By intentionally leaving the definition of visual similarity vague, we hoped to achieve a range of responses, which in overall, would converge to a common sense understanding for similarity. All trials, except for the verification ones, were generated randomly from the dataset during the runtime of the experiment. 

Three of the experiment triplets were used for verification. They contained an obviously similar reference and test image pair to evaluate the reliability of an observer as shown in Figure \ref{fig:verficiation}. As seen from the figure, the test image that is similar to the reference image is another image from the same scene. These images are not used in the actual part of the experiment, and do not belong to the dataset of 100 images shown in Figure \ref{fig:dataset}.

\begin{figure}
\begin{center}
\includegraphics[height=0.85\textheight]{figures/chapter3/verification_small.jpg}
\caption{Verification triplets, shown as 3rd, 10th and 16th trials in the experiment sessions. }
\label{fig:verficiation}
\end{center}
\end{figure}

If an observer failed to provide the correct answer even for one of these trials, his or her data was discarded as being unreliable. These trials were distributed across the experiment to ensure that observers were attentive throughout. Before the experiment began, observers were informed about their task and the expected duration of the experiment, which was at most 20 minutes at a normal pace. During the experiment, observers were required to use the exposure sliders for each image before they made selection. Image selection was done by clicking on one of the test images. The selection was indicated using a green border around the selected image. Observers could change their selection until they pressed the “Next” button. The progress of an observer was indicated using a small progress bar at the bottom center of the screen. At the end of the experiment, observers were informed with a final page confirming the conclusion of the experiment and were presented with unique session ids. They were required to enter this id to the crowdsourcing platform to verify that they have finished the experiment.

\section{Data Collection}
Crowdsourcing has been used in many computer vision problems to collect non-expert data~\cite{kovashka2016crowdsourcing}. In this thesis, in order to reach as many people as possible, the experiment was published at Microworkers crowdsourcing platform\footnote{www.microworkers.com}. For each completed experiment 0.3\$ were paid to the participants.

One of the challenges of data crowdsourcing is eliminating users that give unreliable responses, this may due to not being qualified for the task or just being a spammer~\cite{garcia2016challenges}. To minimize the problem of users being not qualified, the crowd group of English speaking, high qualified users are selected from the crowdsourcing platform. Even though the experiment itself does not require language proficiency, the instructions at the beginning of the experiment is important for users to successfully complete the experiment. The users are in the \emph{high qualified} crowd, if they have been done other crowdsourcing tasks before and got positive feedback. Another measure taken to achieve reliable responses is using verification triplets.

In total, 165 sessions were discarded due to incorrect responses given to the verification trials. Age, gender, and familiarity with computer graphics/image processing distribution of the participants are shown in Figure \ref{fig:age_gender_cgi}. This information was asked to the users at the beginning of the experiment.

\begin{figure}
\begin{center}
\includegraphics[width=\textwidth]{figures/chapter3/age_gender_cgi.pdf}
\caption{Age, gender, and computer graphics/image processing familiarity distribution of the participants.
%paper
}
\label{fig:age_gender_cgi}
\end{center}
\end{figure}

\subsection{Phase I}
\label{sec:exp_phase_I}
In the first phase of the experiment, randomly selected triplets are shown to the users without any restrictions on the image selection. After collecting the experimental results, and eliminating the triplets from invalid sessions, it was found that 18747 unique image triplets were judged by the observers. This amounts to approximately 11.6\% of the total possible triplets that can be obtained from 100 images, $C(100, 3)$. Experiment sessions were independent and random for each participant, but it was guaranteed that a single session consisted of only unique triplets. 

This design resulted in a single response for the majority of the triplets. Some triplets received two responses and only a few received three or more. As such this first phase of the experiment is considered as a random exploration of all possible comparisons. However, as judging similarity based on a single response could be too subjective, the experiment is extended as discussed below to collect multiple responses for each triplet.

\subsection{Phase II}
\label{sec:exp_phase_II}
The first phase of the experiment was extended to obtain three evaluations per triplet. Unlike the first phase where triplets were generated randomly, the second phase solely used the triplets that had been evaluated in the first phase of the experiment. To achieve this, the triplets sorted from the first phase in descending order by the number of responses collected. If a triplet had more than three responses, three of the responses are randomly selected. The triplets with exactly three responses were used as is. These two cases occurred very rarely. Next, triplets with two responses, and then a single response were presented randomly to obtain a total of 4990 triplets that had been evaluated three times. Among these thrice evaluated triplets, 2170 triplet were judged consistently by all three observers. The remaining 2820 triplets generated two-to-one responses. Similar to the first part of the experiment, the second part also contained the same validity checks to eliminate the responses of inattentive observers.











\chapter{Experiment Analysis}
\label{chp:b4}
%paper
Having discussed the details of our crowdsourcing study along with experimented image features, we now explain how we relate features to human judgments. We first present our analysis method for assessing the correlation between each feature type and the experiment results. We then discuss two possible ways to combine the features for developing a more effective similarity model. In our evaluations, we used HDR images directly, as well as by linear scaling, and applying several tone mapping operators. For this purpose, we used the PFStmo software library [32], which provides a reliable implementation of several commonly used TMOs.
%paper
\section{Individual Feature Correlation}
%paper
Assume that $t_i = R_i-A_i-B_i$ represents the $i^{th}$ triplet (i.e. trial) with $R_i$ being the reference image, $A_i$ the left test image, and $B_i$ the right test image. This triplet could have been evaluated one or more times by different observers. Let $n(A_i)$ and $n(B_i)$ represent the number of times that each image was found more similar to $R_i$ than the other. From this information, we created a binary vector to encode the participants’ responses:
\begin{equation}
\label{eq:p_vector}
P = (x_1, \ldots, x_N),
\end{equation}

where each element is defined as:
\begin{equation}
x_i = \begin{cases}
1, \text{if $n(A_i) > n(B_i)$}, \\
0, \text{otherwise}.
\end{cases}
\end{equation}

For each feature type $f$, we also computed the feature representations of each image as $f(R_i)$, $f(A_i)$, $f(B_i)$ and computed their similarity to each other to obtain the following binary vector:
\begin{equation}
    F = (y_1, . . . , y_N ),
\end{equation}
where
\begin{equation}
y_i = \begin{cases}
1, \text{ if $ d(f(R_i), f(A_i)) < d(f(R_i), f(B_i))$ } \\
0, \text{otherwise}.
\end{cases}
\end{equation}

In this equation $d$ represents the distance metric that was chosen to be used for feature $f$. This encoding gave rise to two binary vectors, $P$ and $F$, with the former computed from user responses and the latter from feature similarities. There are many approaches to compute the correlation between two such vectors. We used the Sokal-Michener correlation, which is a simple, intuitive, and effective way to correlate two binary vectors [62]. This correlation is defined as

\begin{equation}
s = {{S_{11}(P, F) + S_{00}(P, F)} \over {N}},
\end{equation}

with $S_{11}$ and $S_00$ representing the total count of matching ones and zeros respectively:

\begin{equation}
    S_{11}(P, F) = P \cdot F,
\end{equation}
\begin{equation}
    S_{00}(P, F) = \neg P \cdot \neg F, 
\end{equation}


Note that the correlation coefficient s can take a value in range [0, 1]. In the following, we multiply this coefficient by 100 to represent the correlations as percentages.

The raw feature correlations with the first (Section 3.3) and the extended experiment (Section 3.4) are reported in Tables \ref{tab:ind_correlation_p1} and \ref{tab:ind_correlation_p2}, respectively. In these tables, the leftmost column indicates the processing type applied to the images before the computation of features.


\begin{sidewaystable}
\caption{Individual feature correlations with the first part of the experiment. The numbers indicate the Sokal-Michener correlation scaled by 100 to represent percentages.}
\centering
\begin{tabular}{r | c c c c c c c}
\label{tab:ind_correlation_p1}
\textbf{Processing Type} & \textbf{VGG16} & \textbf{VGG19} & \textbf{Color} & \textbf{Luminance} & \textbf{Texture} & \textbf{GIST}\\
\hline
HDR-original & 56.79 & 58.09 & 55.10 & 53.14 & 52.39 & 56.82 \\
HDR-linear & 63.54 & 63.31 & 55.69 & 54.07 & 54.36 & 58.18 \\
Drago et al. [10] & 65.88 & 65.74 & 56.73 & 57.45 & 51.17 & 58.23 \\
Mai et al. [31] & 65.28 & 65.13 & 56.01 & 56.77 & 51.90 & 57.57 \\ 
Reinhard et al. (local) [44] & 65.82 & 65.63 & 56.58 & 54.77 & 51.43 & 57.89 \\
Reinhard et al. (global) [44] & 65.75 & 65.52 & 56.59 & 54.68 & 51.39 & 57.92 \\
Durand \& Dorsey [11] & 66.17 & 65.43 & 55.77 & 55.12 & 51.79 & 57.85 \\
Mantiuk et al. [35] & 65.42 & 65.33 & 56.29 & 55.38 & 52.08 & 58.03 \\
Reinhard \& Devlin [43] & 65.28 & 65.20 & 57.15 & 55.89 & 54.85 & 58.33 \\
Fattal et al. [13] & 65.90 & 65.72 & 56.39 & 57.46 & 51.92 & 58.19 \\
Mantiuk et al. [33] & 65.71 & 65.74 & 55.98 & 56.99 & 51.84 & 57.86 \\
Ferradans et al. [14] & 66.02 & 65.90 & 55.18 & 56.51 & 51.99 & 58.33 \\
Pattanaik et al. [41] & 64.46 & 64.38 & 53.04 & 54.61 & 53.06 & 57.84 
\end{tabular}
\end{sidewaystable}


\begin{sidewaystable}
\caption{Individual feature correlations with the second part of the experiment. The numbers indicate the Sokal-Michener correlation scaled by 100 to represent percentages.}
\centering
\begin{tabular}{r|c c c c c c c}
\label{tab:ind_correlation_p2}
\textbf{Processing Type} & \textbf{VGG16} & \textbf{VGG19} & \textbf{Color} & \textbf{Luminance} & \textbf{Texture} & \textbf{GIST}\\
\hline
HDR-original & 64.88 & 67.14 & 60.23 & 58.39 & 54.42 & 63.50 \\
HDR-linear & 75.58 & 76.13 & 60.78 & 57.79 & 57.97 & 65.71 \\
Drago et al. [10] & 80.88 & 81.80 & 62.58 & 62.72 & 53.87 & 65.39 \\
Mai et al. [31] & 80.00 & 79.95 & 61.11 & 61.66 & 53.46 & 64.06 \\
Reinhard et al. (local) [44] & 80.92 & 81.61 & 62.21 & 58.16 & 53.87 & 64.88 \\
Reinhard et al. (global) [44] & 80.92 & 81.57 & 62.21 & 57.97 & 54.75 & 64.75 \\
Durand \& Dorsey [11] & 81.75 & 81.34 & 62.07 & 59.22 & 53.00 & 64.19 \\
Mantiuk et al. [35] & 80.41 & 80.65 & 61.15 & 59.59 & 52.49 & 64.47 \\ 
Reinhard \& Devlin [43] & 80.37 & 80.41 & 64.15 & 61.43 & 60.55 & 65.44 \\
Fattal et al. [13] & 80.51 & 80.92 & 62.30 & 64.24 & 52.90 & 65.02 \\
Mantiuk et al. [33] & 80.00 & 80.78 & 62.12 & 61.71 & 54.56 & 64.19 \\
Ferradans et al. [14] & 81.38 & 82.21 & 58.39 & 61.61 & 55.02 & 65.25 \\
Pattanaik et al. [41] & 78.66 & 78.11 & 57.33 & 58.66 & 55.71 & 64.52
\end{tabular}
\end{sidewaystable}



“HDR-original” represents the unaltered HDR image whereas “HDR-linear” represents its linearly scaled version. The other processing types all include the application of a certain tone mapping operator. For all processing types, except the original, the images were gamma-corrected and scaled to [0, 255] range.
%paper
\section{Combined Feature Correlation}
%paper
Given the individual correlations reported in the previous tables, a natural question that follows is if we can combine them to develop a single objective metric that better correlates with human’s assessment of similarity for HDR images. To this end, we performed two types of linear regression analysis yielding two related
but different models.
%paper
\subsection{Triplet Model}
%paper
In our first analysis, we aimed to develop a model that
predicts which of the two test images is more similar to the reference image using the pairwise distances between the test and reference images. Assuming that j is a feature index, one can compute these pairwise differences as follows:
\begin{equation}
    a_j = d_j(f_j(R), f_j(A)), 
\end{equation}
\begin{equation}
   b_j = d_j(f_j(R), f_j(B)). 
\end{equation}

Here $d_j$ represents the distance metric chosen for the $j^{th}$ feature. The model takes as input these differences for all features (i.e. $j \in {1, 2, 3, 4, 5, 6}$) and computes their weighted average as its response:

\begin{equation}
    r = c_0 + c_1(a_1 - b_1) + c_2(a_2 - b_2) + c_3(a_3 - b_3)+ c_4(a_4 - b_4) + c_5(a_5 - b_5) + c_6(a_6 - b_6)
\end{equation}
 

To compute the unknown coefficients we used logistic regression as our dependent data (i.e. user responses)
were binary: given one reference and two test images, the user selects either the left image or the right one,
encoded as 1 and 0.

The regression was performed between the two vectors, namely the $P$ vector from Equation \ref{eq:p_vector}, and the model response $R$ comprised of the following elements:
\begin{equation}
    R = (r_1,\cdots, r_N ),
\end{equation}
where
\begin{equation}
    r_i = [a_{i1} - b_{i1} \cdots a_{i6} - b_{i6}].
\end{equation}

The logistic regression models the logarithm of the odds as the response of the model:
\begin{equation}
   ln \left( {{Pr(x = 1)} \over { 1 - Pr(x = 1)}} \right) = r. 
\end{equation}


From this equation, it can be derived that the probability of a user responding 1 (i.e. selecting the left image)
is equal to

\begin{equation}
\label{eq:response_prob}
    Pr(x = 1) = {{1} \over {1 + e^{-r}}}
\end{equation}

If we find $Pr(x = 1) > 0.5$, we assume that the model has selected the left image. Otherwise, the model’s response was taken as the right image.

To measure the effectiveness of this model we used 10-fold cross validation. In each fold, 90\% of the trials
were selected for training and the remaining 10\% for testing. This process was repeated 10 times while ensuring that each test fold is mutually exclusive from each other. Similar to the analysis of individual features, we assessed the success of this model against both the original (V1) and the extended experiment (V2). The results are shown in Table \ref{tab:correlation_triplet_model}. It can be seen that the feature combination, on average, improves the success of each presentation type by about 3\% to 4\%. The best three results are obtained by Ferradans et al.’s [14], Drago et al.’s [10], and Reinhard et al.’s [44] TMO algorithms. The reported coefficients are computed by using the entire dataset from the second part of the experiment (V2) due to its higher correlation with the combined features.


\begin{sidewaystable}
\caption{The correlations of the first regression model with the user responses. V1 and V2 represent the first and
extended experiments respectively. The coefficients are reported for the extended experiment only due to its higher correlation with the user data.}
\centering
\begin{tabular}{r|c c || r r r r r r r}
\label{tab:correlation_triplet_model}
\textbf{Processing Type} & \textbf{V1} & \textbf{V2} & \textbf{c0} & \textbf{c1} & \textbf{c2} & \textbf{c3} & 
\textbf{c4} & \textbf{c5} & \textbf{c6}\\
\hline
HDR-original & 60.67 & 70.76 & 0.0573 & 0.0768 & -3.3241 & -0.0028 & -0.0124 & -0.2921 & -10.7505 \\
HDR-linear & 64.81 & 78.83 & 0.0005 & -5.4801 & -5.9902 & -0.0074 & -0.0635 & -0.3289 & -10.1782 \\
Drago et al. [10] & 67.36 & 83.49 & -0.0423 & -7.8751 & -7.6339 & -0.0506 & -0.0958 & 0.0043 & -7.3615 \\
Mai et al. [31] & 66.70 & 81.78 & 0.0085 & -5.2932 & -7.9526 & -0.0601 & -0.1078 & -0.0358 & -4.9275 \\
Reinhard et al. (local) [44] & 67.19 & 83.21 & -0.0154 & -7.3838 & -8.7207 & -0.0688 & -0.0853 & 0.0145 & -7.8380 \\
Reinhard et al. (global) [44] & 67.34 & 83.16 & -0.0230 & -7.0932 & -8.8856 & -0.0687 & -0.0783 & 0.0101 & -7.4470 \\
Durand \& Dorsey [11] & 66.92 & 83.03 & -0.0604 & -8.1694 & -7.3044 & -0.0977 & -0.0147 & 0.0082 & -6.8549 \\
Mantiuk et al. [35] & 66.64 & 81.74 & 0.0220 & -6.1999 & -8.0462 & -0.1081 & -0.0286 & -0.0102 & -10.0494 \\
Reinhard \& Devlin [43] & 66.72 & 82.75 & -0.0332 & -5.6555 & -8.8871 & -0.1284 & -0.0144 & -0.0254 & -7.9970 \\
Fattal et al. [13] & 67.25 & 82.56 & -0.0025 & -6.2320 & -8.3176 & -0.1120 & -0.0272 & -0.0143 & -7.9175 \\
Mantiuk et al. [33] & 66.91 & 82.15 & -0.0005 & -5.7555 & -8.4433 & -0.0777 & -0.0548 & -0.0041 & -6.8189 \\
Ferradans et al. [14] & 67.21 & 83.53 & 0.0226 & -5.7782 & -9.8899 & -0.0801 & -0.0432 & -0.0060 & -7.4090 \\
Pattanaik et al. [41] & 65.02 & 79.89 & -0.0365 & -7.5194 & -5.6052 & 0.0132 & -0.0565 & 0.0211 & -6.1389 \\
\end{tabular}
\end{sidewaystable}



%paper
\subsection{Duplet Model}
Despite the first regression model yielding high correlations exceeding 80\% for most algorithms, it has an important drawback. It requires a triplet of images, one reference and two test, as input to the model. While this matches the presentation type in our experiment, a more desirable model should be able to take only two images (e.g., a query image and a test image) and produce a relative similarity score between them. This may allow, for instance, ranking the similarity of multiple images with a query image as in image-based search applications.

In order to allow for this possibility, our second regression model was designed in the following manner.
For each trial, $ti = R_i - A_i - B_i, i \in {1, . . . , N}$, we inserted two elements to our user response vector:
\begin{equation}
    x_{2i-1} = \begin{cases} 
    1, \text{ if $n(A_i) > n(Bi)$ } \\
    0, \text{otherwise},
    \end{cases}
\end{equation}
\begin{equation}
    x_{2i} = \neg  x_{2i-1},
\end{equation}
    

yielding a vector of size $2N$:
\begin{equation}
   P = (x_1, x_2,\cdots, x_{2N} ). 
\end{equation}

As for the model’s inputs each element of the feature vector was computed as
\begin{equation}
    y_{2i-1} = [a_1 \cdots a_6], 
\end{equation}
\begin{equation}
    y_{2i} = [b_1 \cdots b_6], 
\end{equation}


yielding
\begin{equation}
    F = (y_1, y_2, \cdots , y_{2N} ).
\end{equation}

In summary, the elements of the feature vector always followed the A, B order, whereas the corresponding elements in the user vector were 1 for the selected image and 0 for the other image. This second regression model learns to produce the following response given the feature differences between a reference and test image:

\begin{equation}
\label{eq:log_regression}
    r_a = c_0 + c_1a_1 + c_2a_2 + c_3a_3 + c_4a_4 + c_5a_5 + c_6a_6 
\end{equation}

By converting this response to probability values as in Equation \ref{eq:response_prob}, one can compute a relative degree of similarity between the two images. To validate this model, we computed the model response twice by using $R_i - A_i$ and $R_i - B_i$ image pairs:
\begin{equation}
    Pr(x = left) = {{1} \over {1 + e^{-r_a}}} 
\end{equation}
\begin{equation}
    Pr(x = right) = {{1} \over {1 + e^{-r_b}}}
\end{equation}


Given a triplet, if we found $Pr(x = left) > Pr(x =right)$ we assumed the model to have selected the left image. Otherwise, it was assumed that the model selects the right one. The correlation of this model with the user responses was calculated as in the previous model yielding the results in Table \ref{tab:correlation_duplet_model}. The best result of the second model was found for Drago et al.’s [10] TMO in the extended experiment. The model achieved a correlation of 83.81\% with the user responses.

\begin{sidewaystable}
\caption{The correlations of the second regression model with the user responses. V1 and V2 represent the first
and extended experiments respectively. The coefficients are reported for the extended experiment only due to its
higher correlation with the user data.}
\centering
\begin{tabular}{r|c c || r r r r r r r}
\label{tab:correlation_duplet_model}
\textbf{Processing Type} & \textbf{V1} & \textbf{V2} & \textbf{c0} & \textbf{c1} & \textbf{c2} & \textbf{c3} & 
\textbf{c4} & \textbf{c5} & \textbf{c6}\\
\hline
HDR-original & 60.75 & 70.80 & 2.9323 & -0.1531 & -2.8191 & -0.0024 & -0.0063 & -0.2494 & -7.1569 \\
HDR-linear & 64.65 & 78.50 & 5.5623 & -3.9224 & -3.7111 & -0.0149 & -0.0048 & -0.2164 & -5.5490 \\
Drago et al. [10] & 67.52 & 83.81 & 8.3967 & -5.5248 & -4.0845 & -0.0280 & -0.0587 & -0.0054 & -3.3575 \\
Mai et al. [31] & 66.72 & 81.73 & 7.4594 & -4.0822 & -5.0859 & -0.0196 & -0.0532 & -0.0326 & -0.9064 \\
Reinhard et al. (local) [44] & 67.35 & 83.53 & 8.2123 & -5.6104 & -4.3743 & -0.0249 & -0.0290 & 0.0063 & -3.6705 \\
Reinhard et al. (global) [44] & 67.20 & 83.16 & 8.2162 & -5.3915 & -4.5828 & -0.0259 & -0.0264 & 0.0049 & -3.6673 \\
Durand \& Dorsey [11] & 66.81 & 82.61 & 8.2396 & -5.9298 & -4.0539 & -0.0560 & -0.0081 & 0.0077 & -3.1001 \\
Mantiuk et al. [35] & 66.50 & 82.10 & 7.6833 & -4.3626 & -4.8232 & -0.0658 & -0.0212 & -0.0082 & -3.4889 \\
Reinhard \& Devlin [43] & 66.56 & 82.61 & 8.5676 & -4.6656 & -4.9324 & -0.0936 & -0.0075 & -0.0262 & -2.8843 \\
Fattal et al. [13] & 67.07 & 82.79 & 8.0671 & -4.4119 & -4.8215 & -0.0716 & -0.0191 & -0.0141 & -2.8938 \\
Mantiuk et al. [33] & 66.57 & 82.01 & 7.7805 & -3.9872 & -5.3899 & -0.0228 & -0.0319 & -0.0084 & -2.6625 \\
Ferradans et al. [14] & 67.33 & 83.16 & 8.5911 & -5.0432 & -4.9965 & -0.0541 & -0.0258 & -0.0089 & -2.3825 \\
Pattanaik et al. [41] & 64.94 & 79.84 & 6.6735 & -5.6657 & -3.4601 & 0.0109 & -0.0295 & 0.0241 & -1.8922
\end{tabular}
\end{sidewaystable}

\chapter{Application: Style Based Tonemapping}
\label{chp:b5}

\section{Problem Definition}
\section{Method}
\section{Improvements with User Study Results}
\subsection{Parameter Interpolation with All Features}
\subsection{Parameter Interpolation with Related Features}
\section{Results}

\chapter{Conclusions and Future Work}
\label{chp:b6}
\chapter{Conclusions and Future Work}
\label{chp:b7}

In this thesis, HDR image similarity problem is investigated through an experimental user study and two similarity models are proposed to model this subjective phenomenon. In the user experiment, one reference image and two test images are presented and the participants are asked to choose the test image that is similar to the given reference image. The definition of similarity is not given in order not to constrain the users and introduce bias to the data. The experiment is conducted through a web-based interface which makes it possible to collect the responses of more than 1200 users that are reached through a crowdsourcing platform. To keep the data quality high, several measures are taken such as including verification trials through the experiment.

In order to gain insight about the subjective evaluation of HDR image similarity, the experiment data is analysed with different TMOs, image features and distance metrics. The selected TMOs are heavily used TMOs that prioritize different aspects of tone mapping. The image features include commonly used low level features as well as deep learning features. The distance metrics are chosen in a way that is suitable for the selected image feature representations. Correlations between human judgements and these quantitative features are computed to assess how much each feature contributes to visual similarity. Lastly, two combined features which perform better than individual image features are also proposed with the weights of contributing features are estimated from the user data. 

Reliable assessment of image similarity lies at the hearth of many computer vision applications. In this thesis, an application is given to demonstrate how HDR image similarity can be used for consistently tone mapping various HDR images following a created style. This tone mapping method, namely style based tone mapping, estimates the tone mapping parameters for the given image from the tone mapping parameters of a set of manually tone mapped calibration images using the similarity between the given image and calibration images. The initial basic similarity model of the operator is then improved with two different approaches derived from the findings of the user experiment. The improvements yielded with better tone mapping results in terms of style imitation while keeping the image characteristics intact.

Although a small number of calibration images are used, the tone mapping operator is shown to depict the given style and produce satisfying tone mapping results for HDR images that has different image statistics. It should be also noted that, one of the main benefits of the presented operator is once the calibration images are tone mapped, the new HDR images can be automatically tone mapped with the created style without any intervention such as manual search of optimal parameters. Hence, it makes this operator a suitable choice for the applications where batch tone mapping of HDR images is necessary.

One of the limitations of the proposed tone mapping operator is the performance of the method can be affected by calibration image selection. The operator tone maps similar images similarly, if the input HDR image is too different than the calibration images, the estimated parameters may not give pleasing results. This problem can be prevented by selecting a different set of calibration images for specific domains.

The conducted image similarity experiment also has certain limitations and drawbacks. Firstly, it relies on crowdsourcing, which was necessary to reach a wider audience to collect as much as data possible but made it impossible to control the viewing conditions of the participants. Different results could have been obtained if the experiments were done in a laboratory environment with controlled display and lighting conditions. 

Secondly, the participants compared the HDR images on standard monitors and used sliders to visualize different image regions that are visible on different exposures. Similarly, different results could have been obtained if participants viewed the images on an HDR display. 

Lastly, in the experiment, the meaning of similarity is not given intentionally and left to the interpretation of the participants. To reduce this uncertainty, future studies may explicitly define what is meant by similarity such as object similarity, color similarity, indoor-outdoor similarity or time-of-day similarity.

\section{Future Work}

Although image similarity is an extensively studied topic, HDR image similarity has not gained as much attention yet. To investigate this subjective phenomenon, further experiments which consider ranking and rating tasks as well as pairwise comparisons can be conducted. Also, the proposed models can be extended with different types of features. Evaluations may include DCNNs that are either fine-tuned or trained with HDR data from the ground up.

Given the large number of image quality datasets and subjective evaluations in the form of mean opinions scores (MOS), whether image quality and similarity correlate with each other in the context of HDR imaging can be investigated. Image saliency can also be taken into account for similarity judgements as it was found to improve performance in some other domains~\cite{amirkhani2019inpainted}. Perhaps most importantly, the effect of calibrated HDR images for image similarity and retrieval tasks can be studied. As objects are represented with their true luminances in calibrated data, this may simplify the similarity assessment between the images. Also, with emerging standards for HDR video streaming such as HDR10+ and Dolby Vision, the HDR video similarity problem will gain importance in near future.

Finally, in the recent years, HDR video tone mapping is gaining popularity and more and more studies are conducted to tone map videos coherently~\cite{eilertsen2017comparative}. Style-based tone mapping can be extended for HDR video tone mapping, where the number of images are too many to pick tone mapping parameters manually and often different scenes needs to have the same look of the general style of the video. 


\input{references.tex}

%
% References in Bibtex format goes into below indicated file with .bib extension
%\bibliography{thesis_references}
% You can use full name of authors, however most likely some of the Bibtex entries you will find, will use abbreviated first names
% If you don't want to correct each of them by hand, you can use abbreviated style for all of the references

%\bibliographystyle{abbrv}

% if you have more that one appendix, then use \appendices, otherwise use 
\appendix
\chapter{Clusters of the Hdr Image Dataset}
\label{app:clusters}

\begin{figure}[h!]
\begin{center}
\includegraphics[width=\textwidth]{appendix1/cluster1.png}
\caption{Cluster I, clustering result of Fairchild's HDR Dataset~\cite{fairchild2007hdr} for calibration image selection.}
\end{center}
\end{figure}

\begin{figure}
\begin{center}
\includegraphics[width=\textwidth]{appendix1/cluster2.png}
\caption{Cluster II, clustering result of Fairchild's HDR Dataset~\cite{fairchild2007hdr} for calibration image selection.}
\end{center}
\end{figure}

\begin{figure}
\begin{center}
\includegraphics[width=\textwidth]{appendix1/cluster3.png}
\caption{Cluster III, clustering result of Fairchild's HDR Dataset~\cite{fairchild2007hdr} for calibration image selection.}
\end{center}
\end{figure}

\begin{figure}
\begin{center}
\includegraphics[width=\textwidth]{appendix1/cluster4.png}
\caption{Cluster IV, clustering result of Fairchild's HDR Dataset~\cite{fairchild2007hdr} for calibration image selection.}
\end{center}
\end{figure}

\begin{figure}
\begin{center}
\includegraphics[width=\textwidth]{appendix1/cluster5.png}
\caption{Cluster V, clustering result of Fairchild's HDR Dataset~\cite{fairchild2007hdr} for calibration image selection.}
\end{center}
\end{figure}

\begin{figure}
\begin{center}
\includegraphics[width=\textwidth]{appendix1/cluster6.png}
\caption{Cluster VI, clustering result of Fairchild's HDR Dataset~\cite{fairchild2007hdr} for calibration image selection.}
\end{center}
\end{figure}
\appendix
\chapter{Manually tone mapped calibration images}
\label{app:calib}

\begin{sidewaysfigure}
\begin{center}
\includegraphics[width=\textwidth]{appendix2/calibimg_candy.pdf}
\caption{Manually tone mapped calibration images for \emph{candy} style.}
\label{fig:gallery}
\end{center}
\end{sidewaysfigure}

\begin{sidewaysfigure}
\begin{center}
\includegraphics[width=\textwidth]{appendix2/calibimg_grittily.pdf}
\caption{Manually tone mapped calibration images for \emph{grittily} style.}
\label{fig:gallery}
\end{center}
\end{sidewaysfigure}

\begin{sidewaysfigure}
\begin{center}
\includegraphics[width=\textwidth]{appendix2/calibimg_natural.pdf}
\caption{Manually tone mapped calibration images for \emph{natural} style.}
\label{fig:gallery}
\end{center}
\end{sidewaysfigure}

\begin{sidewaysfigure}
\begin{center}
\includegraphics[width=\textwidth]{appendix2/calibimg_painterly.pdf}
\caption{Manually tone mapped calibration images for \emph{painterly} style.}
\label{fig:gallery}
\end{center}
\end{sidewaysfigure}
\chapter{RESULTS OF STYLE BASED TONE MAPPING}
\label{app:results}
\rotatebox{90}{\begin{minipage}{0.74\textheight}
    \includegraphics[width=\textwidth]{figures/chapter5/style_based/PaulBunyan_hdrcandy_v1_small.jpg}
    \captionof{figure}{High resolution version of Figure~\ref{FigStyle}(a).}
    \label{fig:PropProf}
\end{minipage}}

%\begin{sidewaysfigure}
%\begin{center}
%\includegraphics[width=\textwidth]{figures/chapter5/style_based/PaulBunyan_hdrcandy_v1.png}
%\caption{Manually tone mapped calibration images for \emph{candy} style.}
%\end{center}
%\end{sidewaysfigure}

\begin{sidewaysfigure}
\begin{center}
\includegraphics[width=\textwidth]{figures/chapter5/style_based/PaulBunyan_hdrcandy_v2_small.jpg}
\caption{High resolution version of Figure~\ref{FigStyle}(b).}
\end{center}
\end{sidewaysfigure}

\begin{sidewaysfigure}
\begin{center}
\includegraphics[width=\textwidth]{figures/chapter5/style_based/PaulBunyan_hdrcandy_w0_1_small.jpg}
\caption{High resolution version of Figure~\ref{FigStyle}(c).}
\end{center}
\end{sidewaysfigure}

\begin{sidewaysfigure}
\begin{center}
\includegraphics[width=\textwidth]{figures/chapter5/style_based/PaulBunyan_hdrcandy_w0_w1_w2_small.jpg}
\caption{High resolution version of Figure~\ref{FigStyle}(d).}
\end{center}
\end{sidewaysfigure}

\begin{sidewaysfigure}
\begin{center}
\includegraphics[width=\textwidth]{figures/chapter5/style_based/Peppermill_hdrcandy_v1_small.jpg}
\caption{High resolution version of Figure~\ref{FigStyle}(e).}
\end{center}
\end{sidewaysfigure}

\begin{sidewaysfigure}
\begin{center}
\includegraphics[width=\textwidth]{figures/chapter5/style_based/Peppermill_hdrcandy_v2_small.jpg}
\caption{High resolution version of Figure~\ref{FigStyle}(f).}
\end{center}
\end{sidewaysfigure}

\begin{sidewaysfigure}
\begin{center}
\includegraphics[width=\textwidth]{figures/chapter5/style_based/Peppermill_hdrcandy_w0_1_small.jpg}
\caption{High resolution version of Figure~\ref{FigStyle}(g).}
\end{center}
\end{sidewaysfigure}

\begin{sidewaysfigure}
\begin{center}
\includegraphics[width=\textwidth]{figures/chapter5/style_based/Peppermill_hdrcandy_w0_w1_w2_small.jpg}
\caption{High resolution version of Figure~\ref{FigStyle}(h).}
\end{center}
\end{sidewaysfigure}

\begin{curriculumvitae}
\textbf{PERSONAL INFORMATION}\\


\textbf{Surname, Name: } Aydınlılar, Merve \\
\textbf{Nationality: } Turkish (TC) \\
\textbf{Date and Place of Birth: } 1986, Bursa\\
\textbf{Marital Status: } Single \\

\textbf{EDUCATION} \\
\\
\begin{tabular}{l l l}
    \textbf{Degree} & \textbf{Institution} & \textbf{Year of Graduation}  \\
    M.Sc.  &  Dept. of Computer Engineering, METU & 2011 \\
    B.S. & Dept. of Computer Engineering, METU & 2008 \\
\end{tabular}\\ 
\\
\\
\textbf{PROFESSIONAL EXPERIENCE} \\
\\
\begin{tabular}{l l l}
    \textbf{Year} & \textbf{Place} & \textbf{Enrollment}  \\
    2018 - Present & New Work SE, Hamburg & Data Scientist \\
    2016 - 2018 & Bytro Labs GmbH, Hamburg & Backend Developer \\
    2010 - 2016 & Dept. of Computer Engineering, METU & System Administrator\\
\end{tabular}\\
\\

\textbf{PUBLICATIONS} 
\begin{enumerate}
    \item Ahmet Oguz Akyüz, Kerem Hadimli, Merve Aydinlilar, and Christian Bloch. Style-based tone mapping for HDR images. SIGGRAPH Asia 2013 Technical Briefs, SA 2013.  
    \item Mehmet Akif Akkus, Merve Aydinlilar, and Sinan Kalkan. Range analysis of junctions. Signal Processing and Communications Applications Conference, SIU 2013.
    \item Merve Aydinlilar, Adnan Yazici. Semi-automatic semantic video annotation tool. Computer and Information Sciences III - 27th International Symposium on Computer and Information Sciences, ISCIS 2012.
\end{enumerate}

\end{curriculumvitae}
\end{document}
